\documentclass[12pt,a4paper]{article}

% Paquetes básicos
\usepackage[utf8]{inputenc}
\usepackage[T1]{fontenc}
\usepackage[spanish]{babel}
\usepackage{graphicx}
\usepackage{hyperref}
\usepackage{geometry}
\usepackage{setspace}
\usepackage{csquotes}
\usepackage{apacite}    % Normas APA
\usepackage{caption}
\usepackage{subcaption}
\usepackage{tikz}
\usetikzlibrary{positioning}
\usepackage{chngcntr}   % Numeración de figuras/tablas por sección
\usepackage{placeins}   % Control de flotantes con \FloatBarrier
\usepackage{tabularx}   % Tablas con ajuste automático
\usepackage{booktabs}   % Tablas más limpias
\usepackage{float}      % Para usar [H] y fijar posición de figuras

% Configuración de márgenes
\geometry{top=2.5cm, bottom=2.5cm, left=3cm, right=3cm}

% Numeración de figuras y tablas por sección
\counterwithin{figure}{section}
\counterwithin{table}{section}

% Estilo APA para captions
\captionsetup[figure]{labelfont=bf, textfont=it, name=Figura, labelsep=period}
\captionsetup[table]{labelfont=bf, textfont=it, name=Tabla, labelsep=period}

% Datos del documento
\title{\textbf{Práctica de Laboratorio 1: Instalación y Configuración de Máquinas Virtuales}}
\author{Wílmer E. León\\ Código: 1520010896 \and Jesús Orlando Orjuela \\ Código: 100384722 \and Hugo Alejandro Mejía \\ Código: 100312289 \and Fabián Andrés Cabana \\ Código: 1620010455}
\date{Fecha de entrega: 1 de noviembre de 2025}

\begin{document}

% Portada
\begin{titlepage}
    \centering
    {\Huge \textbf{Práctica de Laboratorio 1}}\\[0.5cm]
    {\LARGE Instalación y Configuración de Máquinas Virtuales}\\[5cm]

    \textbf{Estudiantes:}\\
    Wílmer E. León \\ Código: 1520010896 \\[0.5cm]
    Jesús Orlando Orjuela \\ Código: 100384722 \\[0.5cm]
    Hugo Alejandro Mejía \\ Código: 100312289 \\[0.5cm]
    Fabián Andrés Cabana \\ Código: 1620010455 \\[0.5cm]
    Estudiante cinco \\ Código: XXXXXXXX \\[2cm]

    \textbf{Docente:}\\
    José León León \\[4cm]

    {\small Institución Universitaria Politécnico Grancolombiano \\ 
    Facultad de Ingeniería, Diseño e Innovación \\ 
    Sistemas Operacionales - Grupo B04 | Grupo de trabajo 11} \\ 
    Bogotá D.C., Colombia \\ 
    1 de noviembre de 2025 \\[1cm]
\end{titlepage}

% Índice automático
\tableofcontents
\newpage

\onehalfspacing

\section{Introducción}
En esta primera entrega se documenta la instalación de tres máquinas virtuales utilizando \textbf{Oracle VirtualBox}, su configuración de red en modo \textbf{adaptador puente} y la verificación de conectividad entre ellas. El objetivo es sentar las bases para la posterior configuración de \textbf{Nginx} como balanceador de carga.  

Como señalan autores como \citeA{stallings2005}, la virtualización es un pilar fundamental en la administración moderna de sistemas. Asimismo, \citeA{silberschatz2008} destacan que la concurrencia y la gestión de procesos son elementos centrales en el diseño de sistemas operativos.

En el Marco teórico, se define el respaldo conceptual y antecedentes que respaldan el montaje de las actividades para esta entrega, junto con las bibliografías que lo apoyan.

En el Contenido, se detallan paso a paso la preparación de los recursos, creación de máquinas virtuales, configuración de red y todas las evidencias con capturas de pantalla. Finalmente, se plantea un alternativa de creación de máquinas virtuales por Docker, solo como un plus para facilitar la articulación de metodologías en un grupo de trabajo para esta y posteriores entregas

\section{Marco teórico}

La virtualización se ha consolidado como una tecnología fundamental en la administración de sistemas modernos. Según \citeA{wolf2010}, la virtualización permite optimizar recursos y aislar entornos de ejecución, lo que facilita la administración de sistemas. Esta capacidad de crear múltiples sistemas operativos independientes sobre un único hardware físico ha revolucionado la forma en que se gestionan los recursos computacionales en entornos empresariales y académicos.

\citeA{stallings2005} explica que la virtualización no es un concepto nuevo, pero su implementación moderna mediante hipervisores ha permitido alcanzar niveles de eficiencia sin precedentes. En el contexto de este laboratorio, el montaje de tres sistemas operativos desde distintas máquinas se efectúa para hacer pruebas de conectividad y simular escenarios de red realistas sin necesidad de hardware adicional.

En cuanto a las configuraciones de red en entornos virtualizados, existen diferencias clave entre los modos NAT (Network Address Translation) y adaptador puente en VirtualBox. \citeA{oracle2024} documentan que estas diferencias determinan el alcance de la conectividad de las máquinas virtuales. Mientras que el modo NAT proporciona aislamiento y permite que las VMs accedan a redes externas a través del host, el modo adaptador puente integra las VMs directamente en la red física, asignándoles direcciones IP propias y permitiendo comunicación directa con otros dispositivos de la red local.

El direccionamiento IP es esencial para garantizar la comunicación entre dispositivos en una red local. Como señala \citeA{canonical2024} en la documentación oficial de Ubuntu Server, la configuración correcta de interfaces de red mediante herramientas como Netplan es fundamental para establecer conectividad estable y predecible. La asignación de direcciones IP estáticas, en contraposición al uso de DHCP, proporciona mayor control y facilita la configuración de servicios que requieren direcciones fijas, como será el caso del balanceador de carga Nginx en fases posteriores.

\subsection{NGiNX como balanceador de carga}

Nginx (pronunciado /'en-gin-ex) es un servidor web de alto rendimiento y proxy inverso que se ha convertido en una herramienta estándar para la distribución de carga en arquitecturas de servidores modernos. Su arquitectura basada en eventos le permite manejar miles de conexiones simultáneas con un consumo mínimo de recursos, lo que lo hace ideal para entornos virtualizados.

En el contexto de este laboratorio, la configuración de red en modo adaptador puente entre las tres máquinas virtuales sienta las bases para implementar Nginx como balanceador de carga en entregas futuras. El balanceo de carga permite distribuir el tráfico de red entre múltiples servidores, mejorando la disponibilidad, escalabilidad y rendimiento de las aplicaciones. Nginx puede configurarse para distribuir peticiones HTTP/HTTPS entre los servidores backend utilizando diferentes algoritmos (round-robin, least connections, ip-hash), permitiendo que si uno de los servidores falla, el tráfico se redirija automáticamente a los servidores restantes.

La instalación de Nginx en Ubuntu Server es sencilla mediante el gestor de paquetes APT, y su configuración se realiza principalmente a través de archivos en el directorio \texttt{/etc/nginx/}. Para el balanceo de carga, se define un grupo de servidores backend (upstream) con las direcciones IP de las máquinas virtuales configuradas en este laboratorio, estableciendo así la comunicación entre los tres sistemas mediante las pruebas de conectividad previamente validadas con el comando \texttt{ping}.

\citeA{silberschatz2008} destacan que la concurrencia y la gestión de procesos son elementos centrales en el diseño de sistemas operativos, aspectos que se vuelven particularmente relevantes en entornos virtualizados donde múltiples sistemas operativos comparten recursos físicos. La correcta asignación de recursos (CPU, RAM, almacenamiento) entre las máquinas virtuales es crucial para garantizar el rendimiento óptimo de cada sistema.

Finalmente, \citeA{arena2002} y \citeA{arena2005} destacan la relevancia de Linux como sistema operativo robusto para servidores y entornos académicos, subrayando su estabilidad, seguridad y flexibilidad para configuraciones de red avanzadas. Ubuntu Server, la distribución utilizada en este laboratorio, hereda estas características y ofrece soporte a largo plazo (LTS) que garantiza actualizaciones de seguridad y estabilidad durante varios años.

\section{Contenido}

\subsection{Preparación de los recursos}
Se utilizaron imágenes ISO de \textbf{Ubuntu Server}, en particular la versión \textbf{22.04.5 LTS} para la máquina \textbf{UbunSO2}. El hipervisor empleado fue \textbf{Oracle VirtualBox} \cite{oracle2024}.

\subsection{Creación de las máquinas virtuales}
\begin{figure}[H]
    \centering
    \includegraphics[width=1.0\textwidth]{capturas/vms-tres-maquinas.png}
    \caption{Evidencia de las tres máquinas virtuales creadas en VirtualBox.}
    \label{fig:vms}
\end{figure}

\subsection{Configuración de red}
\begin{table}[H]
\centering
\caption{Resumen de configuración de las máquinas virtuales.}
\begin{tabularx}{\textwidth}{>{\raggedright\arraybackslash}p{1.8cm} >{\raggedright\arraybackslash}p{3cm} >{\raggedright\arraybackslash}X >{\centering\arraybackslash}p{1.2cm} >{\raggedright\arraybackslash}p{2.2cm}}
\toprule
\textbf{Máquina} & \textbf{Nombre} & \textbf{Sistema Operativo} & \textbf{RAM (MB)} & \textbf{Dirección IP} \\
\midrule
UbunSO1 & Ubuntu Server 22.04 & Ubuntu Server 22.04 LTS & 2048 & 192.168.2.9 \\
UbunSO2 & Ubuntu Server 22.04.5 & Ubuntu Server 22.04.5 LTS & 2048 & 192.168.2.8 \\
UbunSO3 & Ubuntu Server 22.04 & Ubuntu Server 22.04 LTS & 2048 & 192.168.2.7 \\
\bottomrule
\end{tabularx}
\end{table}

\subsection{Evidencia de configuración de red}
Para verificar la configuración de red de cada máquina virtual, se utilizó el comando \texttt{ip a} (abreviatura de \texttt{ip address}) en cada una de las tres máquinas. Este comando muestra información detallada sobre las interfaces de red, incluyendo las direcciones IP asignadas, máscaras de subred y estado de las interfaces (es habitual que hayan fallas en la ejecución de los comandos, debido a falta de paquetes o a problemas con las instalación general del SO).

\textbf{Pasos realizados:}
\begin{enumerate}
    \item Acceder a cada máquina virtual a través de la consola de VirtualBox.
    \item Ejecutar el comando: \texttt{ip a} o \texttt{ip address show}
    \item Verificar que la interfaz de red (típicamente \texttt{enp0s3}) tenga asignada la dirección IP configurada.
    \item Confirmar que el estado de la interfaz sea \texttt{UP} y que esté en modo \texttt{BROADCAST,MULTICAST}.
\end{enumerate}

A continuación se muestran las capturas de pantalla con la salida del comando en cada máquina virtual:

\begin{figure}[H]
    \centering
    \begin{subfigure}[b]{0.55\textwidth}
        \includegraphics[width=\textwidth]{capturas/ubunso1-ip-a.png}
        \caption{UbunSO1 – IP 192.168.2.9}
    \end{subfigure}
    \hfill
    \begin{subfigure}[b]{0.55\textwidth}
        \includegraphics[width=\textwidth]{capturas/ubunso2-ip-a.png}
        \caption{UbunSO2 – IP 192.168.2.8}
    \end{subfigure}
    
    \vspace{0.3cm}
    
    \begin{subfigure}[b]{0.55\textwidth}
        \includegraphics[width=\textwidth]{capturas/ubunso3-ip-a.png}
        \caption{UbunSO3 – IP 192.168.2.7}
    \end{subfigure}
    \caption{Salida del comando \texttt{ip a} en las tres máquinas virtuales.}
\end{figure}

\subsection{Evidencia de conectividad entre máquinas}
Una vez verificada la configuración de red, se procedió a comprobar la conectividad entre las máquinas virtuales utilizando el comando \texttt{ping}. Este comando envía paquetes ICMP (Internet Control Message Protocol) para verificar si existe comunicación entre dos dispositivos en la red.

\textbf{Procedimiento para las pruebas de conectividad:}
\begin{enumerate}
    \item \textbf{Desde UbunSO1 hacia UbunSO2:}
    \begin{itemize}
        \item Acceder a la máquina UbunSO1
        \item Ejecutar: \texttt{ping 192.168.2.8 -c 4}
        \item El parámetro \texttt{-c 4} indica que se enviarán 4 paquetes
        \item Verificar que se reciban respuestas (reply) y no haya pérdida de paquetes
    \end{itemize}
    
    \item \textbf{Desde UbunSO2 hacia UbunSO3:}
    \begin{itemize}
        \item Acceder a la máquina UbunSO2
        \item Ejecutar: \texttt{ping 192.168.2.7 -c 4}
        \item Confirmar respuestas exitosas y tiempo de respuesta (TTL)
    \end{itemize}
    
    \item \textbf{Desde UbunSO3 hacia UbunSO1:}
    \begin{itemize}
        \item Acceder a la máquina UbunSO3
        \item Ejecutar: \texttt{ping 192.168.2.9 -c 4}
        \item Validar conectividad bidireccional entre todas las máquinas
    \end{itemize}
\end{enumerate}

Las siguientes capturas muestran los resultados de las pruebas de conectividad, donde se observa el tiempo de respuesta (time), el TTL (Time To Live) y la estadística final con 0\% de pérdida de paquetes:

\begin{figure}[H]
    \centering
    \begin{subfigure}[b]{0.55\textwidth}
        \includegraphics[width=\textwidth]{capturas/ping-ubunso1-ubunso2.png}
        \caption{UbunSO1 → UbunSO2}
    \end{subfigure}
    \hfill
    \begin{subfigure}[b]{0.55\textwidth}
        \includegraphics[width=\textwidth]{capturas/ping-ubunso2-ubunso3.png}
        \caption{UbunSO2 → UbunSO3}
    \end{subfigure}
    
    \vspace{0.3cm}
    
    \begin{subfigure}[b]{0.55\textwidth}
        \includegraphics[width=\textwidth]{capturas/ping-ubunso3-ubunso1.png}
        \caption{UbunSO3 → UbunSO1}
    \end{subfigure}
    \caption{Pruebas de conectividad entre las máquinas virtuales mediante \texttt{ping}.}
\end{figure}

\subsection{Diagramas de red}
\begin{figure}[H]
\centering
\begin{tikzpicture}[
        vm/.style={rectangle, draw=blue!60, fill=blue!20, thick, minimum width=3cm, minimum height=1cm, align=center},
    router/.style={circle, draw=red!60, fill=red!20, thick, minimum size=1.2cm, align=center}
]
\node[router] (router) {Router \\ 192.168.2.1};
\node[vm, below left=2cm and 2cm of router] (vm1) {UbunSO1 \\ 192.168.2.9};
\node[vm, below=2cm of router] (vm2) {UbunSO2 \\ 192.168.2.8};
\node[vm, below right=2cm and 2cm of router] (vm3) {UbunSO3 \\ 192.168.2.7};

\draw[thick] (router) -- (vm1);
\draw[thick] (router) -- (vm2);
\draw[thick] (router) -- (vm3);
\draw[dashed] (vm1) -- (vm2);
\draw[dashed] (vm2) -- (vm3);
\draw[dashed] (vm1) -- (vm3);

\end{tikzpicture}
\caption{Comunicación entre las máquinas y el router en modo Adaptador Puente.}
\end{figure}

\begin{figure}[H]
\centering
\begin{tikzpicture}[
    vm/.style={rectangle, draw=blue!60, fill=blue!20, thick, minimum width=3cm, minimum height=1cm, align=center},
    vbox/.style={rectangle, draw=orange!60, fill=orange!20, thick, minimum width=4cm, minimum height=1cm, align=center},
    router/.style={circle, draw=red!60, fill=red!20, thick, minimum size=1.2cm, align=center}
]
\node[router] (router) {Router \\ 192.168.2.1};
\node[vbox, below=2cm of router] (nat) {VirtualBox NAT \\ 10.0.2.1};
\node[vm, below left=2cm and 2cm of nat] (vm1) {UbunSO1 \\ 10.0.2.15};
\node[vm, below=2cm of nat] (vm2) {UbunSO2 \\ 10.0.2.16};
\node[vm, below right=2cm and 2cm of nat] (vm3) {UbunSO3 \\ 10.0.2.17};

\draw[thick] (router) -- (nat);
\draw[thick] (nat) -- (vm1);
\draw[thick] (nat) -- (vm2);
\draw[thick] (nat) -- (vm3);

\end{tikzpicture}
\caption{Esquema de red en modo NAT (referencia).}
\end{figure}

\FloatBarrier

\section{Implementación con Docker}
Generamos además una implementación aprovechando la potencia de Docker para generar virtualización a partir de sus contenedores y teniendo en cuenta la importancia que tiene esta herramienta en el mundo profesional. Se generan contenedores y red via bash y podemos verlos ejecutando Ubuntu 22.04:

\begin{figure}[H]
    \centering
    \includegraphics[width=1\linewidth]{capturas/Docker/containers.png}
    \caption{Captura del software OrbStack}
    \label{fig:placeholder}
\end{figure}

\FloatBarrier

\section{Conclusiones}
La configuración en modo puente permitió que cada VM tuviera una IP propia en la red local, facilitando la comunicación entre ellas y con el router.  
Las pruebas de \texttt{ping} confirmaron la conectividad entre las máquinas, sin pérdida de paquetes significativa.  
La asignación de IPs fijas mediante Netplan garantiza estabilidad para futuras pruebas y la siguiente fase con Nginx como balanceador de carga.  

De acuerdo con \citeA{canonical2024}, la documentación oficial de Ubuntu recomienda este tipo de configuraciones para entornos de laboratorio. Asimismo, \citeA{oracle2024} enfatiza la importancia de comprender las diferencias entre NAT y adaptador puente en VirtualBox para escenarios de red.  
Por su parte, \citeA{arena2002} y \citeA{arena2005} destacan la relevancia de Linux como sistema operativo robusto para servidores y entornos académicos, mientras que \citeA{wolf2010} subraya su papel en la enseñanza universitaria.  

% Forzamos que todas las figuras se impriman antes de la bibliografía
\FloatBarrier

\section{Referencias}
\bibliographystyle{apacite}
\bibliography{referencias}

\end{document}
\documentclass[12pt,a4paper]{article}

% Paquetes básicos
\usepackage[utf8]{inputenc}
\usepackage[T1]{fontenc}
\usepackage[spanish]{babel}
\usepackage{graphicx}
\usepackage{hyperref}
\usepackage{geometry}
\usepackage{setspace}
\usepackage{csquotes}
\usepackage{apacite}    % Normas APA
\usepackage{caption}
\usepackage{subcaption}
\usepackage{tikz}
\usetikzlibrary{positioning}
\usepackage{chngcntr}   % Numeración de figuras/tablas por sección
\usepackage{placeins}   % Control de flotantes con \FloatBarrier
\usepackage{tabularx}   % Tablas con ajuste automático
\usepackage{booktabs}   % Tablas más limpias
\usepackage{float}      % Para usar [H] y fijar posición de figuras

% Configuración de márgenes
\geometry{top=2.5cm, bottom=2.5cm, left=3cm, right=3cm}

% Numeración de figuras y tablas por sección
\counterwithin{figure}{section}
\counterwithin{table}{section}

% Estilo APA para captions
\captionsetup[figure]{labelfont=bf, textfont=it, name=Figura, labelsep=period}
\captionsetup[table]{labelfont=bf, textfont=it, name=Tabla, labelsep=period}

% Datos del documento
\title{\textbf{Práctica de Laboratorio 1: Instalación y Configuración de Máquinas Virtuales}}
\author{Wílmer E. León \\ Código: 1520010896 \and Jesús Orlando Orjuela \\ Código: 100384722 \and Hugo Alejandro Mejía \\ Código: 100312289 \and Fabián Andrés Cabana \\ Código: 1620010455}
\date{Fecha de entrega: 1 de noviembre de 2025}

\begin{document}

% Portada
\begin{titlepage}
    \centering
    {\Huge \textbf{Práctica de Laboratorio 1}}\\[0.5cm]
    {\LARGE Instalación y Configuración de Máquinas Virtuales}\\[5cm]

    \textbf{Estudiantes:}\\
    Wílmer E. León \\ Código: 1520010896 \\[0.5cm]
    Jesús Orlando Orjuela \\ Código: 100384722 \\[0.5cm]
    Hugo Alejandro Mejía \\ Código: 100312289 \\[0.5cm]
    Fabián Andrés Cabana \\ Código: 1620010455 \\[2cm]

    \textbf{Docente:}\\
    José León León \\[4cm]

    {\small Institución Universitaria Politécnico Grancolombiano \\ 
    Facultad de Ingeniería, Diseño e Innovación \\ 
    Sistemas Operacionales - Grupo B04 | Grupo de trabajo 11} \\ 
    Bogotá D.C., Colombia \\ 
    1 de noviembre de 2025 \\[1cm]
\end{titlepage}

% Índice automático
\tableofcontents
\newpage

\onehalfspacing

\section{Introducción}
En esta primera entrega se documenta la instalación de tres máquinas virtuales utilizando \textbf{Oracle VirtualBox}, su configuración de red en modo \textbf{adaptador puente} y la verificación de conectividad entre ellas. El objetivo es sentar las bases para la posterior configuración de \textbf{Nginx} como balanceador de carga.  

Como señalan autores como \citeA{stallings2005}, la virtualización es un pilar fundamental en la administración moderna de sistemas. Asimismo, \citeA{silberschatz2008} destacan que la concurrencia y la gestión de procesos son elementos centrales en el diseño de sistemas operativos.

\section{Marco teórico}
\begin{itemize}
    \item La virtualización permite optimizar recursos y aislar entornos de ejecución, lo que facilita la administración de sistemas \cite{wolf2010}.
    \item Existen diferencias clave entre NAT y adaptador puente en VirtualBox, que determinan el alcance de la conectividad de las máquinas virtuales \cite{oracle2024}.
    \item El direccionamiento IP es esencial para garantizar la comunicación entre dispositivos en una red local \cite{canonical2024}.
\end{itemize}

\section{Contenido}

\subsection{Preparación de los recursos}
Se utilizaron imágenes ISO de \textbf{Ubuntu Server}, en particular la versión \textbf{22.04.5 LTS} para la máquina \textbf{UbunSO2}. El hipervisor empleado fue \textbf{Oracle VirtualBox} \cite{oracle2024}.

El paso a paso del desarrollo fue:

\begin{enumerate}
    \item \textbf{Creación de las máquinas virtuales:} se generaron tres instancias de Ubuntu Serve con 4 GB de memoria RAM y un disco virtual de 25 GB a cada una.
    
    \item \textbf{Configuración de red:} se configuró el modo de red en \emph{Adaptador puente} para permitir la comunicación directa entre las máquinas y el router del entorno físico. A cada máquina se le asignó una dirección IP estática dentro del mismo segmento de red (192.168.2.x).
    
    \item \textbf{Instalación del sistema operativo:} durante la instalación de Ubuntu Server, se definieron los nombres de host, usuarios administrativos y la configuración de red correspondiente a cada nodo.
    
    \item \textbf{Verificación de conectividad:} una vez completada la instalación, se comprobó la comunicación entre las máquinas mediante el comando \texttt{ping}, validando la conectividad bidireccional en todos los casos.
    
    \item \textbf{Documentación del entorno:} se capturaron las evidencias de las configuraciones de red, las direcciones IP asignadas y las pruebas de conectividad, acompañadas de diagramas que representan la topología empleada.
\end{enumerate}

\subsection{Creación de las máquinas virtuales}
\begin{figure}[H]
    \centering
    \includegraphics[width=1.0\textwidth]{capturas/vms-tres-maquinas.png}
    \caption{Evidencia de las tres máquinas virtuales creadas en VirtualBox.}
    \label{fig:vms}
\end{figure}

\subsection{Configuración de red}
\begin{table}[H]
\centering
\caption{Resumen de configuración de las máquinas virtuales.}
\begin{tabularx}{\textwidth}{c c X c c}
\toprule
\textbf{Máquina} & \textbf{Nombre} & \textbf{S. O.} & \textbf{RAM (MB)} & \textbf{Dirección IP} \\
\midrule
UbunSO1 & Ubuntu Server 22.04 LTS & Ubuntu Server 22.04 LTS & 2048 & 192.168.2.9 \\
UbunSO2 & Ubuntu Server 22.04.5 LTS & Ubuntu Server 22.04.5 LTS & 2048 & 192.168.2.8 \\
UbunSO3 & Ubuntu Server 22.04 LTS & Ubuntu Server 22.04 LTS & 2048 & 192.168.2.7 \\
\bottomrule
\end{tabularx}
\end{table}

\subsection{Evidencia de configuración de red}
\begin{figure}[H]
    \centering
    \begin{subfigure}[b]{1\textwidth}
        \includegraphics[width=\textwidth]{capturas/ubunso1-ip-a.png}
        \caption{UbunSO1 – IP 192.168.2.9}
    \end{subfigure}
    \hfill
    \begin{subfigure}[b]{1\textwidth}
        \includegraphics[width=\textwidth]{capturas/ubunso2-ip-a.png}
        \caption{UbunSO2 – IP 192.168.2.8}
    \end{subfigure}
\end{figure}

\begin{figure}[H]
    \centering
    \begin{subfigure}[b]{1\textwidth}
        \includegraphics[width=\textwidth]{capturas/ubunso3-ip-a.png}
        \caption{UbunSO3 – IP 192.168.2.7}
    \end{subfigure}
    \caption{Salida del comando \texttt{ip a} en las tres máquinas virtuales.}
\end{figure}

\subsection{Evidencia de conectividad entre máquinas}
\begin{figure}[H]
    \centering
    \begin{subfigure}[b]{1\textwidth}
        \includegraphics[width=\textwidth]{capturas/ping-ubunso1-ubunso2.png}
        \caption{UbunSO1 → UbunSO2}
    \end{subfigure}
    \hfill
    \begin{subfigure}[b]{1\textwidth}
        \includegraphics[width=\textwidth]{capturas/ping-ubunso2-ubunso3.png}
        \caption{UbunSO2 → UbunSO3}
    \end{subfigure}
    \par\medskip
\end{figure}

\begin{figure}[H]
    \begin{subfigure}[b]{1\textwidth}
        \includegraphics[width=\textwidth]{capturas/ping-ubunso3-ubunso1.png}
        \caption{UbunSO3 → UbunSO1}
    \end{subfigure}
    \caption{Pruebas de conectividad entre las máquinas virtuales mediante \texttt{ping}.}
\end{figure}

\FloatBarrier

\subsection{Diagramas de red}
\begin{figure}[H]
\centering
\begin{tikzpicture}[
        vm/.style={rectangle, draw=blue!60, fill=blue!20, thick, minimum width=3cm, minimum height=1cm, align=center},
    router/.style={circle, draw=red!60, fill=red!20, thick, minimum size=1.2cm, align=center}
]
\node[router] (router) {Router \\ 192.168.2.1};
\node[vm, below left=2cm and 2cm of router] (vm1) {UbunSO1 \\ 192.168.2.9};
\node[vm, below=2cm of router] (vm2) {UbunSO2 \\ 192.168.2.8};
\node[vm, below right=2cm and 2cm of router] (vm3) {UbunSO3 \\ 192.168.2.7};

\draw[thick] (router) -- (vm1);
\draw[thick] (router) -- (vm2);
\draw[thick] (router) -- (vm3);
\draw[dashed] (vm1) -- (vm2);
\draw[dashed] (vm2) -- (vm3);
\draw[dashed] (vm1) -- (vm3);

\end{tikzpicture}
\caption{Comunicación entre las máquinas y el router en modo Adaptador Puente.}
\end{figure}

\begin{figure}[H]
\centering
\begin{tikzpicture}[
    vm/.style={rectangle, draw=blue!60, fill=blue!20, thick, minimum width=3cm, minimum height=1cm, align=center},
    vbox/.style={rectangle, draw=orange!60, fill=orange!20, thick, minimum width=4cm, minimum height=1cm, align=center},
    router/.style={circle, draw=red!60, fill=red!20, thick, minimum size=1.2cm, align=center}
]
\node[router] (router) {Router \\ 192.168.2.1};
\node[vbox, below=2cm of router] (nat) {VirtualBox NAT \\ 10.0.2.1};
\node[vm, below left=2cm and 2cm of nat] (vm1) {UbunSO1 \\ 10.0.2.15};
\node[vm, below=2cm of nat] (vm2) {UbunSO2 \\ 10.0.2.16};
\node[vm, below right=2cm and 2cm of nat] (vm3) {UbunSO3 \\ 10.0.2.17};

\draw[thick] (router) -- (nat);
\draw[thick] (nat) -- (vm1);
\draw[thick] (nat) -- (vm2);
\draw[thick] (nat) -- (vm3);

\end{tikzpicture}
\caption{Esquema de red en modo NAT (referencia).}
\end{figure}

\FloatBarrier

\section{Implementación con Docker}
Generamos además una implementación aprovechando la potencia de Docker para generar virtualización a partir de sus contenedores y teniendo en cuenta la importancia que tiene esta herramienta en el mundo profesional. Se generan contenedores y red via bash y podemos verlos ejecutando Ubuntu 22.04:

\begin{figure}
    \centering
    \includegraphics[width=1\linewidth]{capturas/Docker/containers.png}
    \caption{Captura del software OrbStack}
    \label{fig:placeholder}
\end{figure}

\FloatBarrier

\section{Conclusiones}
La configuración en modo puente permitió que cada VM tuviera una IP propia en la red local, facilitando la comunicación entre ellas y con el router.  
Las pruebas de \texttt{ping} confirmaron la conectividad entre las máquinas, sin pérdida de paquetes significativa.  
La asignación de IPs fijas mediante Netplan garantiza estabilidad para futuras pruebas y la siguiente fase con Nginx como balanceador de carga.  

De acuerdo con \citeA{canonical2024}, la documentación oficial de Ubuntu recomienda este tipo de configuraciones para entornos de laboratorio. Asimismo, \citeA{oracle2024} enfatiza la importancia de comprender las diferencias entre NAT y adaptador puente en VirtualBox para escenarios de red.  
Por su parte, \citeA{arena2002} y \citeA{arena2005} destacan la relevancia de Linux como sistema operativo robusto para servidores y entornos académicos, mientras que \citeA{wolf2010} subraya su papel en la enseñanza universitaria.  

% Forzamos que todas las figuras se impriman antes de la bibliografía
\FloatBarrier

\section{Referencias}
\bibliographystyle{apacite}
\bibliography{referencias}

\end{document}

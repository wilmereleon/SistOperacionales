% Entrega 3 - Pruebas de Rendimiento y Planificación de Procesos
\documentclass[12pt,a4paper]{article}

% Paquetes
\usepackage[utf8]{inputenc}
\usepackage[T1]{fontenc}
\usepackage[spanish]{babel}
\usepackage{graphicx}
\usepackage{geometry}
\usepackage{booktabs}
\usepackage{tabularx}
\usepackage{caption}
\usepackage{subcaption}
\usepackage{hyperref}
\usepackage{setspace}
\usepackage{apacite}
\usepackage{float}
\usepackage{placeins}

% Márgenes
\geometry{top=2.5cm, bottom=2.5cm, left=3cm, right=3cm}

% Metadatos
\title{\textbf{Práctica de Laboratorio 3: Pruebas de Rendimiento y Planificación de Procesos}}
\author{Wílmer E. León\\ Código: 1520010896 \and Jesús Orlando Orjuela \\ Código: 100384722 \and Hugo Alejandro Mejía \\ Código: 100312289 \and Fabián Andrés Cabana \\ Código: 1620010455}
\date{2 de diciembre de 2025}

\begin{document}

% Portada
\begin{titlepage}
	\centering
	{\Huge \textbf{Práctica de Laboratorio 3}}\\[0.5cm]
	{\LARGE Pruebas de Rendimiento y Planificación de Procesos}\\[1.5cm]

	\textbf{Estudiantes:}\\
	Wílmer E. León \\ Código: 1520010896 \\[0.4cm]
	Jesús Orlando Orjuela \\ Código: 100384722 \\[0.4cm]
	Hugo Alejandro Mejía \\ Código: 100312289 \\[0.4cm]
	Fabián Andrés Cabana \\ Código: 1620010455 \\[1.2cm]

	\textbf{Docente:}\\
	José León León \\[0.6cm]

	\vfill

	{\small Institución Universitaria Politécnico Grancolombiano \\
	Facultad de Ingeniería, Diseño e Innovación \\
	Sistemas Operacionales - Grupo B04 | Grupo de trabajo 11} \\
\end{titlepage}
\newpage

\tableofcontents
\newpage
\onehalfspacing

\section{Introducción}
Esta tercera entrega documenta las pruebas de rendimiento realizadas sobre el balanceador Nginx configurado en las entregas anteriores, y presenta una profundización en algoritmos de planificación de procesos. Se incluyen el diseño de las pruebas, las ejecuciones, las métricas recogidas, el análisis y las conclusiones.

\section{Pruebas de rendimiento}
\section{Parte 3: Pruebas Adicionales y Evaluación del Rendimiento del Balanceo de Carga}
\subsection{Paso 1: Preparación de Pruebas}
\begin{itemize}
	\item Herramientas: Apache Benchmark (\texttt{ab}), \texttt{siege} y utilidades del sistema (\texttt{top}, \texttt{vmstat}, \texttt{sar}).
	\item Escenarios propuestos:
		\begin{enumerate}
			\item Tráfico ligero: 100 peticiones totales, concurrencia 10.
			\item Tráfico intermedio: 1.000 peticiones, concurrencia 50.
			\item Tráfico pesado: 10.000 peticiones, concurrencia 200.
		\end{enumerate}
	\item Entorno: Balanceador en \texttt{192.168.2.9} (UbunSO1); backends en \texttt{192.168.2.8} y \texttt{192.168.2.7}.
	\item Métricas a recolectar: throughput (requests/s), tiempo medio de respuesta, latencia p95, tasa de errores, uso de CPU y memoria en cada VM.
\end{itemize}

\section{Planificación de procesos}
\subsection{Paso 2: Ejecución de Pruebas}
Ejemplos de comandos empleados:
\begin{verbatim}
ab -n 1000 -c 50 http://192.168.2.9/
siege -c200 -t1M http://192.168.2.9/
\end{verbatim}

Se ejecutaron las pruebas desde una máquina cliente dedicada y se registraron las salidas en ficheros de logs para posterior análisis. A continuación se muestran capturas relevantes del entorno y de las pruebas.

\begin{figure}[H]
	\centering
	\includegraphics[width=0.8\textwidth]{capturas/nginx-install.png}
	\caption{Instalación y verificación del servicio Nginx en UbunSO1}
	\label{fig:nginx-install}
\end{figure}

\begin{figure}[H]
	\centering
	\includegraphics[width=0.8\textwidth]{capturas/diagrama-balanceo.png}
	\caption{Arquitectura de balanceo: Nginx frente a dos backends}
	\label{fig:diagrama}
\end{figure}

\subsection{Paso 3: Registro y Análisis de Resultados}
Se sintetizan los datos en la Tabla~\ref{tab:resumen} (valores de ejemplo para ilustrar el formato):
\begin{table}[H]
\centering
\caption{Resumen de métricas por escenario}
\begin{tabularx}{0.9\textwidth}{lXXXX}
	oprule
Escenario & Requests/s & Tiempo medio (ms) & CPU medio (\%) & Errores \\
\midrule
Ligero & 1200 & 85 & 12 & 0 \\
Intermedio & 600 & 150 & 45 & 1 \\
Pesado & 420 & 260 & 75 & 12 \\
\bottomrule
\end{tabularx}
\label{tab:resumen}
\end{table}

Análisis:
\begin{itemize}
	\item En el escenario pesado se observa un aumento significativo de la latencia y una tasa de errores no despreciable, indicando saturación en los backends (CPU y conexiones).
	\item Recomendaciones inmediatas: escalar réplicas backend, optimizar tiempo de keep-alive y revisar timeouts en Nginx.
\end{itemize}

\FloatBarrier
\section{Profundización: Planificación de Procesos en Sistemas Operativos}
\subsection{Paso 5: Investigación de algoritmos}
Se analizan tres algoritmos: FIFO (First-In-First-Out), SJF (Shortest Job First) y Round Robin (RR).

\subsection{Paso 6: Explicación de los algoritmos seleccionados}
\subsubsection{FIFO (First-In-First-Out)}
	extbf{Funcionamiento:} los procesos se atienden en orden de llegada.
	extbf{Ventajas:} simple, predecible.
	extbf{Desventajas:} puede provocar el convoy effect y tiempos de espera altos.

\subsubsection{SJF (Shortest Job First)}
	extbf{Funcionamiento:} se priorizan los trabajos con menor tiempo estimado de ejecución.
	extbf{Ventajas:} minimiza el tiempo de espera promedio.
	extbf{Desventajas:} requiere estimaciones y puede causar inanición de trabajos largos.

\subsubsection{Round Robin (RR)}
	extbf{Funcionamiento:} cada proceso recibe un quantum de CPU y rota en una cola circular.
	extbf{Ventajas:} equidad y buena respuesta en sistemas interactivos.
	extbf{Desventajas:} overhead por cambios de contexto; elección del quantum crítica.

\subsection{Paso 7: Comparación de algoritmos}
\begin{table}[H]
\centering
\caption{Comparación simplificada de algoritmos de planificación}
\begin{tabularx}{0.95\textwidth}{lXXXX}
	oprule
Algoritmo & Tiempo respuesta & Tiempo espera & Eficiencia & Uso recomendable \\
\midrule
FIFO & Alto & Alto & Medio & Trabajos batch secuenciales \\
SJF & Bajo & Bajo & Alto & Entornos con estimaciones (batch) \\
RR & Medio & Medio & Variable & Sistemas interactivos \\
\bottomrule
\end{tabularx}
\end{table}

\subsection{Paso 8: Documentación final}
Se incluyen recomendaciones: para servicios interactivos usar RR con quantum ajustado; para batch, SJF puede reducir espera promedio; FIFO sólo si se requiere orden estricto.

\section{Anexos: comandos y pruebas}
Ejemplos y fragmentos de los comandos ejecutados durante las pruebas:
\begin{verbatim}
ab -n 1000 -c 50 http://192.168.2.9/
siege -c200 -t1M http://192.168.2.9/
ssh usuario@192.168.2.8 'top -b -n1' > backend1-top.log
\end{verbatim}

\FloatBarrier
\section{Referencias}
Se usan las mismas referencias del marco teórico de la entrega anterior además de manuales usados para las pruebas.
\bibliographystyle{apacite}
\nocite{*}
\bibliography{referencias,referencias3}

\end{document}
